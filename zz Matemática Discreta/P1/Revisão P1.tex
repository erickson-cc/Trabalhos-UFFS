\documentclass[]{article}
\usepackage[]{amsmath}
\usepackage[]{amsfonts}
\title{Revisão Prova 1\\Matemática Discreta\\Neri}
\author{Erickson G. Müller}

\begin{document}
	\maketitle
	\section*{Conteúdos}
		\begin{enumerate}
			\item Proposições logicamente equivalentes
			\item Lógica proposicional
			\item Argumentos válidos, argumentos verbais
			\item Regras de inferência
			\item Lógica de predicados
			\item Quantificadores universal e existencial
			\item Regras de inferência para quantificadores
			\item Técnicas de demonstração: direta, contraposição, exaustão e absurdo
			\item Teoria dos conjuntos, subconjuntos
			\item Álgebra dos conjuntos
			\item Relações: binárias, equivalência
			\item Partições
			\item Funções: domínio e imagem
			\item Funções: injetora, sobrejetora, bijetora
			\item Composição de funções
			\item Função inversa
		\end{enumerate}
	\newpage
	A matemática pode ser dividida em dois \textbf{domínios}: o \textit{contínuo} e o \textit{discreto}. A matemática contínua estuda conceitos infinitos em seu objetivo, utilizando o sistema de números reais. A matemática discreta utiliza um domínio de números que não estão conectados da mesma forma que os números reais. É uma comparação semelhante à diferença entre o sinal analógico e o digital.\\
	
	A matemática discreta exige do aluno que sejam desenvolvidas demonstrações (provas), para isso existem diversos \textbf{esquemas de provas} que se aplicam a cada caso. O autor do livro recomenda elaborar as provas escrevendo a primeira e a última sentença, e ir desenvolvendo em direção ao meio até que ambas se encontrem.
\newpage
	\section{Correção da Prova do Semestre Passado}
		\subsection{}Para provar que qualquer intervalo fechado $[a,b]$ é equipotente a $\mathbb{R}$, podemos seguir o roteiro abaixo:
		\begin{enumerate}
			\item Prove que qualquer intervalo fechado $[a,b]$ é equipotente ao intervalo $[0,1]$. Isto pode ser feito mostrando que $f:[0,1] \to [a,b]$ onde $f(x) = xb+(1-x)a$ é uma bijeção.
			\item A seguir, prove que o intervalo $[0,1]$ e $\mathbb{R}$ são equipotentes.
		\end{enumerate}
		\begin{itemize}
			\item Faça o item 1.
			\item Explique por que o roteiro acima é suficiente.
			\item Mostre que é uma bijeção a função $g: [0,1]\to \mathbb{R}$ dada por:\\
			$g(x) = \dfrac{1}{2}$, se $x=0$\\
			$g(x) = \dfrac{1}{n+2}$, se existe $n \in \mathbb{N}$ tal que $x = \dfrac{1}{n}$\\
	$g(x) = x$, se $x \notin \{\dfrac{1}{n}|n>0\} \bigcup \{0\}$
		\end{itemize}
		\subsection{} Agora que você sabe fazer a questão 4 da lista, é fácil explira por que qualquer intervalo fechado não é enumerável, já que $\mathbb{R}$ não é enumerável. Mas e \textbf{mostrar que qualquer intervalo fechado é um conjunto infinito}, você sabe fazer?. Se conseguir escrever uma demonstração clara, ganha mais um ponto e meio.
		\subsection{} A questão 3 da lista pedia para você demonstrar que se $A$ é enumerável e $B$ é finito, então $A \bigcup B$ também é enumerável. Mas se $B$ não for enumerável, pode acontecer de $A \bigcup B$ não ser enumerável. Mostre um exemplo em que isto ocorre. Lempre que você tem que convencer que o conjunto $A \bigcap B$ não é enumerável.
		\subsection{}Prove por indução que se $\#A = n$, então $P(A) =2^n$.(Tem no livro do Menezes).
		\newpage
		\subsection{}Considere a sequência dada por recorrência ao lado:\\
		$a0 = $ número incógnito\\
		$a1 = $ número incógnito\\
		$a2 = $ número incógnito\\
		$a3 = $ número incógnito\\
		$an = 8a_{n-2} - 16a_{n-4}+3^n$ se $n \geq 4$\\
		\begin{itemize}
			\item Encontre uma solução particular para a recorrência desta sequência. (lempre, ela tem que ter a foram $a_n = a.3^n$.
%			\item Eu s
		\end{itemize}
	\section{Respostas}
	\begin{itemize}
		\item 1.a\\
			Demonstrar bijeção em\\
			$f:[0,1] \to [a,b]$\\			
			$_{x \to f(x) =xb+(1-x)a}$\\
%			Provar que f é injetora e sobrejetora
			$$f(x) = xb+(1-x)a$$
			$$f(x) = xb +xa +a$$
			$$f(x) = x(b-a)+a$$
			Demonstrar injetividade\\
			Suponha que $f(m)=f(n)$\\
			então:
			$$f(m)=m(b-a)+a$$
			$$f(n) =n(b-a)+a$$
			$$f(m)=f(n) =>$$
			$$m(b-a)+a = n(b-a)+a$$
			$$m=n$$
			Logo, se $f(m)=f(n)$, então $m=n$.\\
			Assim, $f$ é injetora.\\
			Demonstrar sobrejetividade\\
			Nessa situação, precisamos encontrar um $x \in [0,1]$ para cada $y \in [a,b]$ de tal forma que $f(x) = y$\\
			Assim, considere $y \in [a,b]$.\\
			Tome:\\
			$y = x.(b-a)+a$\\
			$y-a = x.(b-a)$
			$$x = \dfrac{y-a}{b-a}$$
			Observe que $x$ de fato pertence ao interval $[0,1]$, pois se $y =a, x=0$ e se $y=b, x=1$\\
			e se $a<y<b$, então\\
			$$y-a < b-a$$
			e $\dfrac{y-a}{b-a}<1$
			$$0<\dfrac{y-a}{b-a}<1 \to o<x<1$$
			$$x \in [0,1]$$
			e Então, se $x = \dfrac{y-a}{b-a}$\\
			$f(x) = x(b-a) + a$\\
			$\dfrac{y-a}{b-a}.(b-a) +a$\\
			$y-a +a = y$\\
			$y=y$\\
			Assim, $f$ é bijetora.
		\item 1.b\\
			Demonstrar bijeção entre $\mathbb{R} \to [0,1]$ e entre $[0,1] \to [a,b]$\\
			Vamos supor que:\\
			$\varphi: [a,									 \to [0,1]$ e $\psi: [0,1] \to \mathbb{R}$ sejam bijeções.\\
			Então a função composta $\psi o \varphi : [a,b] \to \mathbb{R}$ são equipotentes.
		\item 2\\
			Existem 4 tipos de intervalos fechados:\\
			$[a,b] = \{x \int \mathbb{R}/a\leq x \leq b\}$\\
			$[a,\infty) = \{x \int \mathbb{R}/a\leq x \}$\\
			$(-\infty ,b] = \{x \int \mathbb{R}/x \leq b\}$\\
			$(-\infty,\infty) = \mathbb{R}$\\
			Um conjunto $A$ é infinito quando existe uma bijeção\\
			$\varphi B \to A$ onde $B  \subset A$ e $B \not = A$
	\end{itemize}

\end{document}