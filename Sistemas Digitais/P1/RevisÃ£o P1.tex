\documentclass[•]{article}
\title{Revisão P1\\Prof. Geomar}
\date{26 de Setembro de 2024}
\author{Erickson G. Müller}

\begin{document}
	\maketitle
	\section*{Conteúdo}
		\begin{enumerate}
			\item Circuitos Combinacionais
			\item Circuito Contador Binário
			\item Circuito Contador de Gray
			\item Linguagem de Descrição de Hardware VHDL
		\end{enumerate}
	\newpage
	
	\section{Sistema Digital}
		Um aparato dotado de conjuntos finitos de \textbf{entradas} e \textbf{saídas} e capaz de processar informação representada sob forma \textbf{numérica}.
	\section{Circuitos Combinacionais}
	\section{Níveis de Abstração}
		Em ordem decrescente:
		\begin{enumerate}
			\item Nível de Sistema: CPU, ASIP, ASIC, barramentos, memórias, software embarcado.
			\item Nível RT (Transferência entre Registradores): Unidades funcionais (somadores, subtratores, multiplicadores), Rede de interconexão (fios, multiplexadores, decodificadores, barramentos, buffers tri-state), Registradores e blocos de memória RAM, ROM.
			\item Nível Lógico: portas lógicas, latches e flip-flops.
			\item Nível de Circuito Elétrico: transistores, resistores, capacitores, indutores e fios.
			\item Nível de Transistor.
		\end{enumerate}
\end{document}
