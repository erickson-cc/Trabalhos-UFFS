\documentclass[ ]{article}

\usepackage[ ]{amsmath}

\title{Revisão P2 de Probabilidade e Estatística \\ Leandro Bordin}
\author{Erickson G. Müller}
\date{19 de Junho de 2024}

\begin{document}

\maketitle
\section{Conteúdo}
\begin{enumerate}
	\item Estimação de Parâmetros
\end{enumerate}
\pagebreak

\section{Teoria da Estimação para uma Amostra}
	Existem dois tipos de dados para representar a amostra: Estimativa Pontual e Estimativa Intervalar. Como a variabilidade amostral pode resultar estimativas diferentes conforme as amostras selecionadas, agrega-se uma estimativa intervalar para acompanhar a estimativa pontual.

\subsection{Teorema do Limite Central}	
	A variabilidade amostral se	 comporta como uma distribuição normal para amostras maiores ou iguais a 30.

\subsection{Fórmulas da Estimativa}
	\begin{equation*}
		Estimativa Pontual: ux = \overline{x}
	\end{equation*}
	\begin{equation*}
		Estimativa Intervalar: ux = \overline{x} +- z.\dfrac{desvio padrao}{\sqrt[]{n}}
	\end{equation*}	
\subsection{Fórmulas da Proporção}
	\begin{equation*}
		Estimativa Pontual da Proporcao: p = \overline{p} = \dfrac{x}{n}
	\end{equation*}
	\begin{equation*}
		Estimativa Intervalar da Proporcao: p = \overline{p} +- \sqrt[]{\dfrac{\overline{p}.(1-\overline{p})}{n}}
	\end{equation*}
\end{document}

