\documentclass{article}
\usepackage[]{amsmath}
\usepackage{amssymb}
\usepackage[english]{babel}
\usepackage{tikz}
\usepackage[]{cancel}

\title{Revisão Prova 1 de Cálculo II\\Milton Kist}
\author{Erickson Giesel Müller}
\begin{document}
	\maketitle
	
	\section{Conteúdos}
		\begin{enumerate}
			\item Integrais primitivas
			\item Integrais indefinidas
			\item Métodos de integração: Substituição e Integração por partes.
			\item Integração definida via somas de Riemann
			\item Teorema Fundamental do Cálculo
			\item Integração envolvendo funções trigonométricas
			\item Técnicas de integração
			\item Cálculo de áreas de figuras planas
			\item Cálculo de comprimento de arco de curva plana
			\item Cálculo de volumes de sólidos de revolução
			\item Cálculo de áreas de superfícies de revolução
		\end{enumerate}
	\newpage
	\section{Integrais Primitivas}
		Uma função $F(x)$ é considerada primitiva de $f(x)$ em um intervalo I se para todo $x \in I$ temos $F'(x)=f(x)$.
		\subsection{Propriedades}
			$$G'(x)=(F(x)+k)' = F'(x)+0=F'(x)$$
			
		
\end{document}