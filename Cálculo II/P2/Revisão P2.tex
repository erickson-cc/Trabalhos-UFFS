\documentclass{article}
\usepackage[]{amsmath}
\usepackage{pgfplots}
\pgfplotsset{compat=1.18}
\usepackage{amssymb}
\usepackage[]{pxfonts}
\usepackage[english]{babel}
\usepackage{tikz}
\usepackage[]{cancel}

\title{Revisão Prova 2 de Cálculo II\\Milton Kist}
\author{Erickson Giesel Müller}
\date{2 de Dezembro de 2024}
\begin{document}
	\maketitle
	
	\section*{Conteúdos}
		\begin{enumerate}
			\item Funções de várias variáveis (Definição, Domínio, Imagem, Operações, Representação Gráfica).
			\item Limite e Continuidade.
			\item Limite e Continuidade de funções de várias variáveis.
			\item Limites por caminho.
			\item Cálculo de Limites envolvendo indeterminações.
			\item Verificação de Continuidade de funções.
			\item Derivadas parciais e aplicações.
			\item Gradiente.
			\item Multiplicadores de Lagranje.
			\item Integração dupla.
			\item Integração tripla
			
		\end{enumerate}
	\newpage
	
	\section{Função de Várias Variáveis}
		Seja $A\subset \mathbb{R}^n$, a relação $f_i A \to \mathbb{R}$ é denominada função real, $P=(x_1,x_2,...,x_n), P\in \mathbb{R}$, associamos um único número real $z \in \mathbb{R}$.
		$$A=D(f)$$
		$$\mathbb{R}=CD(f)$$
		$$Im(f)=\{z \in \mathbb{R}/z=f(x_1,x_2,...)\}$$
		\textbf{Exemplo:} Dada a função $f(x,y)=\sqrt[]{1-x^2-y^2}$, determine os conjuntos domínio e imagem de $f$.
		$$1-x^2-y^2 \geq 0 \leftrightarrow x^2+y^2 < 1$$
		$$D(f) = \{(x,y) \in \mathbb{R}^2/x^2+y^2\leq 1\}$$
		$$Im(f)=[0,1]$$
		
		\textbf{Exemplo:} em cada caso, determine o domínio da função, faça também a representação geométrica do domínio:
	
	%aula que eu faltei abaixo
	\section{Regra da Cadeia}
		Vamos definir a regra da cadeia para o caso de funções de várias variáveis.\\
		Proposição 1: Sejam A e B conjuntos abertos do $\mathbb{R}^2$ e $\mathbb{R}$, respectivamente, e sejam $f:A \to \mathbb{R}$ e $g:B \to \mathbb{R}^2$ tais que $g(t) = (x(t),y(t)) \in A$ para todo $t \in B$. Nestas condições, se $g$ for diferenciável em $B$ e $f(x,y)$ possuir derivadas parciais de 1ª ordem contínuas em $A$.\\
		Então, a função composta:\\
			$$h(t)= f(g(t))=f(x(t),y(t))$$
		é diferenciável $\forall t \in B$ e $\dfrac{dh}{dt}$ é dada por:
			$$\dfrac{dh}{dt} = \dfrac{\varphi f}{\varphi x}. \dfrac{dx}{dt}+\dfrac{\varphi f}{\varphi y}.\dfrac{dy}{dt}$$
		Exemplo: sejam $h(t)=f(x(t),y(t))$, $f(x,y)=5xy-y^2$ e $x(t) = 2t$, $y(t)=5t+3$\\
		Verificar que neste caso vale (*). Usando (*)\\
		$$\dfrac{dh}{dt} = 5y.2+(5x-2y).5$$
		$$=5.(5t+3).2+(5(2t)-2(5t+3)).5$$
		$$=50.t+30+(10t-10t-6).5 = 50t$$
		$$\to h(t) = f(2t,5t+3)$$
		$$=5.(2t)(5t+3)-(5t+3)^2$$
		$$=50t^2+30t-25t^2-30t-9$$
		$$= 25t^2-9$$
		$$h'(t) = 50t$$
		confere\\
		\textbf{Proposição 2:} Sejam A e B conjuntos abertos em $\mathbb{R}^2$, e sejam $z=f(u,v)$, $u=u(x,y)$ e $v=v(x,y)$ funções diferenciáveis, então a função composta $h(x,y) = f(u(x,y),v(x,y))$ é diferenciável.
		$$\dfrac{\varphi h}{\varphi x} = \dfrac{\varphi f}{\varphi u} . \dfrac{\varphi u}{\varphi x} +\dfrac{\varphi f}{\varphi v}. \dfrac{\varphi v}{\varphi x}$$
		$$\dfrac{\varphi h}{\varphi y} = \dfrac{\varphi f}{\varphi u}.\dfrac{\varphi u}{\varphi y}+\dfrac{\varphi f}{\varphi v}.\dfrac{\varphi v}{\varphi y}$$
		%Tinha algo aqui que terei que conferir no livro
		\textbf{Exemplo:} Sejam $f(u,v) =2u^3-3v+3	2$, $w(x,y)=2x-y$ e $v(x,y)=2x - xy^2$. Determine $\dfrac{\varphi f}{\varphi x}$ e $\dfrac{\varphi f}{\varphi y}$:
		$$\to \dfrac{\varphi f}{\varphi x}(x,y)=\dfrac{\varphi f}{\varphi u}.\dfrac{\varphi u}{\varphi x} + \dfrac{\varphi f}{\varphi v}. \dfrac{\varphi v}{\varphi x}$$	
		$$= 6u^2.2-5.(2-y^2)= 12.(2x-y)^2-5.(2-y^2)$$		
		
		$$\to \dfrac{\varphi f}{\varphi y}(x,y)= \dfrac{\varphi f}{\varphi u}.\dfrac{\varphi u}{\varphi y}+ \dfrac{\varphi f}{\varphi v}.\dfrac{\varphi v}{\varphi y}$$
		$$= 6u^2.(-1)-5.(-2xy)= -6.(2x-y)^2+10xy$$
		
		%Ver no livro a generalização da regra da cadeia
	\section{Derivadas Parciais Sucessivas}
		Seja $z=f(x,y)$, já sabemos como construir as funções $\dfrac{\varphi f}{\varphi x}, \dfrac{\varphi f}{\varphi y}$. Da mesma forma, podemos construir as funções:
		$$\dfrac{\varphi ^2 f}{\varphi x^2} = \dfrac{\varphi}{\varphi x}.(\dfrac{\varphi f}{\varphi x})$$
		$$\dfrac{\varphi^2f}{\varphi y . \varphi x}= \dfrac{\varphi}{\varphi y}.(\dfrac{\varphi f}{\varphi x})$$
		$$\dfrac{\varphi ^3f}{\varphi y^2.\varphi x} = \dfrac{\varphi}{\varphi y}.(\dfrac{\varphi}{\varphi y}.(\dfrac{\varphi f}{\varphi x}))$$
		\textbf{Exemplo:} Dada a função $f(x,y) = 5x^2y-y^3x^2+9$. Determine as seguintes funções derivadas:
		\begin{enumerate}
			\item $\dfrac{\varphi^2f}{\varphi x^2}$
			\item $\dfrac{\varphi ^2 f}{\varphi y \varphi x}$
			\item $\dfrac{\varphi ^3f}{\varphi y^2\varphi x}$
		\end{enumerate}
		1.
		$$\dfrac{\varphi f}{\varphi x} = 10xy-2y^3x$$
		$$\dfrac{\varphi ^2f}{\varphi x^2} = \dfrac{\varphi}{\varphi x}.(\dfrac{\varphi f}{\varphi x}) = 10y - 2y^3$$
		2.
		$$\dfrac{\varphi ^2f}{\varphi y \varphi x}=\dfrac{\varphi}{\varphi y}.(\dfrac{\varphi f}{\varphi x})=10-6xy^2$$
		3.
		$$\dfrac{\varphi ^3 f}{\varphi y^2 \varphi x}=\dfrac{\varphi}{\varphi y}.(\dfrac{\varphi}{\varphi y}.(\dfrac{\varphi f}{\varphi x})) = -12xy$$
		
		
	\section{Teorema de Schwartz}
		Seja $f:A \subset \mathbb{R}^2\to \mathbb{R}$, $A$ aberto. Se $f$ for de classe $C^2$ em $A$, então:
		$$\dfrac{\varphi ^2 f}{\varphi x \varphi y}(x,y) = \dfrac{\varphi ^2 f}{\varphi y \varphi x}(x,y), \forall (x,y) \in A$$
		\textbf{Exemplo:} Dada a função $f(x,y)= 4x^3y^4+y^2$.\\
		Verifique se $\dfrac{\varphi ^2 f}{\varphi x \varphi y}(x,y) = \dfrac{\varphi ^2 f}{\varphi y \varphi x}$:
		$$\dfrac{\varphi f}{\varphi x} = 12x^2.y^4$$
		$$\dfrac{\varphi ^2 f}{\varphi y \varphi x} =\dfrac{\varphi}{\varphi y} .(\dfrac{\varphi f}{\varphi x}) = 48 x^2y^3$$
		ou
		$$\dfrac{\varphi f}{\varphi y} = 16 x^3y^3+2y$$
		$$\dfrac{\varphi ^2 f}{\varphi x \varphi y} = \dfrac{\varphi}{\varphi x}.(\dfrac{\varphi f}{\varphi y}) = 48x^2y^3$$
	\section{Máximos e Mínimos de Funções de Duas Variáveis}
		\textbf{Definição:} Seja $z=f(x,y)$ uma função de duas variáveis, então:
		\begin{enumerate}
			\item $(x_0,y_0) \in D(f)$ é um ponto de \textbf{máximo local} de $f$, se existir uma bola aberta $B = B((x_0,y_0),r)$ tal que $f(x,y) \leq f(x_0,y_0) , \forall (x,y) \in B \bigcap D(f)$
		\end{enumerate}
		
	\section{Teorema de Weierstrass} 
		Seja $A \subset \mathbb{R}^2$ um domínio fechado e limitado, $f:A \to \mathbb{R}$ definida por $z=f(x,y)$ uma função contínua em A. Então existem $P_1$, $P_2 \in A$ tais que $f(P_1) \leq f(P) \leq f(P_2), \forall P \in A$\\
		\textbf{Exemplo:} Seja dada a função $f(x,y)=y^3-x^2y+4y$. Determine o valor de máximo e o valor de mínimo de f sobre o conjunto A delimitado pelo triângulo de vértices $M(-2,0)$, $N(0,2)$, $0(0,-2)$  \textit{FIGURA 1}.
		\textbf{Observação:}
		\begin{itemize}
			\item Pelo Teorema de Weierstrass, f admite máximo e mínimo global sobre A.
			\item Para determinarmos os candidatos a extremos máximo e mínimo, vamos analisar separadamente o interior de A e a fronteira de A.
			\item Os candidatos a extremos que estão localizados no interior de A estão entre os pontos críticos da função. Isto é: onde $\dfrac{\varphi f}{\varphi x}(x,y)=0$ e $\dfrac{\varphi f}{\varphi y}(x,y)=0$ ou onde essas derivadas parciais não existem.
		\end{itemize}
		Determinar os candidatos a extremos que estão no interior de A.
		$$\dfrac{\varphi f}{\varphi x}(x,y)=-2xy$$
		$$\dfrac{\varphi f}{\varphi y}(x,y)= 3y^2-x^2+4$$
		logo\\
		$\dfrac{\varphi f}{\varphi x}=0$ e $\dfrac{\varphi f}{\varphi y}= 0$\\
		$-2xy=0$ e $3y^2 -x^2+4=0$\\
		$$-2xy=0 \to xy=0 $$
		logo $ x=0$ ou $y=0$
		se $x=0 \to 3y^{2-0^2}+4=0$, $\nexists y \in \mathbb{R}$\\
		se $y=0 \to 3.0^2-x^2+4=0\to x= \pm 2$\\%ajustar
		Pontos críticos (candidatos a extremos): $(-2,0),(2,0)$.\\
		Nenhum dos dois pontos está no interior de A. Logo não são candidatos a extremos.\\
		Determinar os candidatos a extremos que estão na fronteira de A.
		$$\overline{MN} = (-2,0),(0,2) \to y = ax+b$$
		$$0 = a.(-2) +b$$
		$$-2a+2=0\to a = 1$$
		e
		$$2=a.0+b \to b=2$$
		$$y=x+2, -2\leq x \leq 0$$
		
		$$\overline{MO}=(-2,0),(0,-2)$$
		$$0=a(-2)+b \to -2a - 2 = 0 $$
		e
		$$-2 = a.0+b \to b = -2 \to a = -1$$
		$$y = -x-2, -2\leq x \leq 0$$
		
		Sobre o segmento $\overline{MN}: y=x², -2 \leq x \leq 0$:\\
		substituindo na função:
		$$f(x,y) = f(x,x+2)$$
		$$= (x+2)^3-x^2.(x+2)+4(x+2)$$
		$$= x^3+6x^2+12x+8-x^3-2x^2+4x+8$$%(a+b)³=a³+3a²b+3ab²+b³
		$$=4x^2+16x+16, -2\leq x \leq 0$$% o x varia de -2 até 0
		Para determinarmos os candidatos a extremos sobre $\overline{MN}$ usamos os conceitos de Cálculo I.
		$$f_1(x)=4x^2+16x+16$$
		$$f'_1(x)=8x+16=0 \to 8x+16=0 \to x=-2$$
		$x=-2$ não é ponto interior do segmento $\overline{MN}$. Logo os pontos críticos sobre o segmento $\overline{MN}$ não serão apenas seus extremos $x=-2$ e $x=0$.
		\textbf{Candidatos a extremos:} (-2,0), (0,2)\\
		quando $x= -2$ e $y=0$.\\
		quando $x=0$ e $y=2$.\\
		$$\overline{ON}: x=0, -2 \leq y \leq 2$$
		$$\overline{MN}: y= x+2$$
		\textbf{Segmento $\overline{ON}$:} $x=0, -2 \leq y \leq -2$\\
		Substituindo na função f.
		$$f(0,y) = y^3-0^2.y+4y = y^3+4y$$
		$$f_2(y) = y^3+4y$$
		$$f'_2(y) = 3y^2 +4 = 0 $$
		$\nexists y$ pois são dois quadrados e não tem como ser menor que 0.\\
		Logo, neste segmento, os candidatos a extremos serão: $(0,-2), (0,2)$
		\textbf{Segmento $\overline{MO}$:}$y=-x-2,-2 \leq x \leq 0$ \\
		Substituindo na função f:\\
		$$f(x,-x-2)=(-x-2)^3-x^2(-x-2)+4(-x-2)$$
		$$=-x^3-6x^2-12x-8+x^3+2x^2-4x-8$$
		$$=4x^2-16x-16 = f_3(x)$$
		$$f'_3(x)=-8x-16=0 \to 8x+16 = 0 \to x=-2$$
		que está no extremo.\\
		São candidatos a extremos sobre $\overline{MO}$: $(-2,0),(0,-2)$\\
		Quadro de Análise \textit{FIGURA 2}\\
		
		\textbf{Exemplo:} A temperatura T em qualquer ponto $(x,y)$ do plano é dada por $T(x,y) = 3y^2+x^2-x$. Qual é a temperatura máxima e mínima sobre o disco fechado de raio 2 centrado na origem do sistema ortogonal cartesiano $p=(0,0)$?\\
		Região A = $\{(x,y) \in \mathbb{R}^2; x^2+y^2 \leq 4\}$\textit{FIGURA 3}\\
		\textbf{Interior de A: }\\
		Pontos Críticos\\
		$$\dfrac{\varphi f}{\varphi x}(x,y)=2x-1=0 \to x=\dfrac{1}{2}$$%derivada de f com relação a x
		$$\dfrac{\varphi f}{\varphi y}(x,y)=6y=0 \to y=0$$
		$(\dfrac{1}{2},0)$ está no interior de A.\\
		\textbf{Fronteira de A:}\\
		Pontos Críticos\\
		$$x^2+y^2 =2^2$$
		$$y^2=4-x^2, -2 \leq x \leq 2$$
		Substituindo na função f:\\
		$$T(x,y) = 3.(4-x^2)+x^2-x$$
		$$= 12-3x^2+x^2-x = -2x^2-x+12$$
		$$T_1(x) = -2x^2 -x +12$$
		$$T'_1(x) = -4x-1=0 \to -4x-1=0 \to x = -\dfrac{1}{4}$$
		Se $x = -\dfrac{1}{4}\to y^2 = 4-(\dfrac{-1}{4})^2= 4-\dfrac{1}{16}= \dfrac{63}{16}$
		$$y = \dfrac{\pm \sqrt[]{63}}{4}$$
		Sobre a fronteira de A, candidatos a extremos:
			\begin{enumerate}
				\item $(-\dfrac{1}{4},\dfrac{\sqrt[]{63}}{4})$
				\item $(-\dfrac{1}{4},-\dfrac{\sqrt[]{63}}{4})$
				\item $ (-2,0)$
				\item $(2,0)$
			\end{enumerate}
		extremos do intervalo \textit{-2 e 2}, $x=2\to y^2=0$, $x=2 \to y^2 =0 \to y = 0$\\
		Quadro de Análise \textit{FIGURA 4}
		
	\section{Máximos e Mínimos Condicionados}
		Considere os problemas:
		\begin{itemize}
			\item min $f(x,y) = x^2+y^2$ \textit{problema de otimização irrestrito/livre}
			\item min $f(x,y) = x^2 +y^2$\\
			sujeito a $x+y=1$  \textit{problema de otimização restrita/condicional}
		\end{itemize}
		\textit{FIGURA 5}]\\
		Problemas envolvendo Funções de duas variáveis e uma restrição:\\
		Considere o problema:\\
		max $f(x,y)$\\
		sujeito a $g(x,y) = 0$\\
		
		\textbf{Proposição}: Seja $f(x,y)$ uma função diferenciável sobre um domínio U. Seja $g(x,y)$ uma função com derivadas parciais contínuas em U. Tal que o gradiente de $g(x,y)$ não pode ser vetor nulo ($\nabla g(x,y) \neq (0,0)$). Gradiente é o vetor formado pelas derivadas parciais.\\
		$\forall (x,y) \in V$, onde $V = \{(x,y) \in U , g(x,y)=0\}$\\
		Uma condição necessária para que $(x_0,y_0) \in U$ seja extremo local de f em V é que:\\
		$$\nabla f(x_0,y_0) = \lambda \nabla g(x_0,y_0)$$
		para algum $\lambda \in \mathbb{R}$\\
		\textbf{OBS:} Podemos dizer que os pontos de máximo ou de mínimo condicionados de $f$ devem satisfazer as seguintes condições:
		$$g(x,y) = 0, \dfrac{\varphi f}{\varphi x} (x,y) = \lambda \dfrac{\varphi g}{\varphi x}(x,y)$$
		e (*)
		$$\dfrac{\varphi f}{\varphi y}(x,y) = \lambda \dfrac{\varphi g}{\varphi y} (x,y)$$
		para algum $\lambda$ real.\\
		\textbf{OBS:} O $\lambda \in \mathbb{R}$ que torna (*) compatível é denominado de multiplicador de Lagranje.
		O método de Lagranje consiste em definir uma função de 3 variáveis.
		$$L(x,y,\lambda)=f(x,y)-\lambda g(x,y)$$
		Nesse caso a equação (*) é equivalente à equação gradiente $\nabla L (x,y,\lambda)=0$\\
		Equação gradiente:
		$$\nabla L (x,y,\lambda) = ( \dfrac{\varphi L}{\varphi x}, \dfrac{\varphi L}{\varphi y}, \dfrac{\varphi L}{\varphi \lambda})=(\dfrac{\varphi f}{\varphi x}-\lambda \dfrac{\varphi g}{\varphi x}, \dfrac{\varphi f}{\varphi y}- \lambda \dfrac{\varphi g}{\varphi y}, g(x,y))=(0,0,0)$$
		Com isso os candidatos a extremos de $f$ sujeitos à restrição $g(x,y) = 0$ são exatamente os pontos críticos de L.\\
		\textbf{Exemplo:} Uma casa retangular deve ser construída em um terreno triangular de forma que a casa tenha área máxima. Se o terreno tiver o formato \textit{FIGURA 6}, quais deverão ser as dimensões da casa?\\
		Dimensões do terreno: Triângulo reto de base 30 e altura 10.\\
		\textbf{Solução:}Vamos colocar a situação problema num sistema ortogonal cartesiano. \textit{FIGURA 7}.//
		\textbf{Problema modelado:}\\
		max $A(x,y) = x.y$\\
		sujeito a $ 3y+x-30=0$\\
		$\to$ Equação de Lagranje
		$$L(x,y,\lambda)=xy-\lambda(3y+x-30)$$
		$$\nabla L(x,y,\lambda) = (y-\lambda,x-3\lambda,-(3y+x-30))=0$$
		$\nabla L = 0 \to $
		\begin{itemize}
			\item $y-\lambda =0 \to y= \lambda$
			\item $x-3\lambda = 0 \to x = 3\lambda$
			\item $3y+3x-30 = 0 \to 3\lambda+3\lambda-30 = 0$
		\end{itemize}
		logo\\
		$6\lambda =30 \to \lambda = 5 \to x = 15$ e $y=5$\\
		Logo a solução do problema será:
		$$A(15,5) = 15.5=75m^2$$
		\textbf{Analogamente}, podemos ter problemas da forma:\\
		max/min $f(x,y,z)$\\
		sujeito a $g(x,y,z)=0$\\
		Neste caso a equação de Lagrange será:
		$$L(x,y,z,\lambda) = f(x,y,z) - \lambda g(x,y,z)$$
 \end{document}