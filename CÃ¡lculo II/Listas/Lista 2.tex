\documentclass{article}
\usepackage{amsmath}
\usepackage{amsfonts}
\usepackage{graphicx} % Required for inserting images

\title{Cálculo II\\Lista 2 - Funções de Várias Variáveis}
\author{Erickson G. Müller}
\date{}

\begin{document}
\maketitle
\section*{Lista 1.4 (pg. 26)}
	\begin{itemize}
		\item 2
			\\$D_A=1300-50x+20y$
			\\$D_B=1700+12x-20y$
			\\$R_x=x.D_A$
			\\$R_y=y.D_B$
			\\$R_T=x.D_A+y.D_B$
			$$R_T = 1300x-50x^2+20xy+1700y+12xy-20y^2$$
			$$R_T = 32xy-50x^2-20y^2+1300x+1700y$$
		\item 3.a \\$z=3-x-y$ %é um plano inclinado infinito
			$$D(z)=\mathbb{R}^2$$% todo o plano xy
			$$Im(z) = \mathbb{R}$$ %toda a linha z
		\item 3.b \\$f(x,y) = 1+x^2+y^2$
			$$D(f(x,y)) = \mathbb{R}^2$$
			$$Im(f(x,y)) = [1,\infty)$$
		
		\item 3.c \\$z=\sqrt[ ]{9-(x^2+y^2)}$
		\\$x^2+y^2\leq 3^2$
		\\$x^2+y^2-9 \leq 0$
		$$D(z)=\{(x,y) \in \mathbb{R}^2 / x^2+y^2 \leq 9\}$$
		\\$\sqrt[•]{9-(x^2+y^2)} \to \sqrt[]{9-9}= 0$
		\\$x^2+y^2 \geq 0 \to \sqrt[]{9-0} = 3$
		$$Im(z) = [0,3]$$
		\item 3.d
		\\$w = e^{x^2+y^2+z^2}$
		$$D(w) = \mathbb{R}^3 $$		
		$$Im(w) = [1, \infty)$$
		\item 3.j
		\\$f(x,y) = 4-x^2-y^2$
		$$D(f(x,y)) = \mathbb{R}^2$$
		$$Im(f(x,y)) = (-\infty,4]$$
		\item 4.b
		\\$w = \dfrac{1}{x^2+y^2+z^2}$
		\\$x^2+y^2+z^2 \geq 0$
		$$D(w) = \{(x,y,z) \in \mathbb{R}^3 / (x,y,z) \neq (0,0,0)\}$$
		$$Im(w) = (0, \infty)$$
		\item 4.c 
		\\$z=\dfrac{1}{\sqrt[•]{x^2-y^2}}$
		\\$x^2-y^2 > 0$
		$$D(z) = \{(x,y) \in \mathbb{R}^2 / |x|>|y|\}$$
		$$Im(z) = (0,\infty)$$
		\item 4.e
		\\$z=\sqrt[•]{x^2+y^2-1}$
		\\$x^2+y^2-1 \geq 0$
		$$D(z) = \{(x,y) \in \mathbb{R}^2 / x^2+y^1 \geq 1\}$$
		$$Im(z) = [0,\infty)$$
		\item 4.i 
		\\$y = \sqrt[•]{\dfrac{1+x}{1+z}}$
		\\$1+z \neq 0$ \& $\dfrac{1+x}{1+z}\geq 0$
		\\$z\neq-1$
		\\$x \geq -1$ se $z > -1$
		\\$x \leq -1$ se $z < -1$
		\begin{align*}
		D(y) = \{(x,z) \in \mathbb{R}^2 /  x &\geq -1 se z>-1\\
		x&\leq -1 se z<-1\}
		\end{align*}
		\item 4.j
		\\$w = \dfrac{1}{9-x^2-y^2-z^2}$
		\\$9-x^2-y^2-z^2 \neq 0$
		\\$x^2+y^2+z^2 \neq 9$
		$$D(w) = \{(x,y,z) \in \mathbb{R}^3 / x^2+y^2+z^2 \neq 9\}$$		
		\item 4.n
		\\$z = \ln (x+y-3)$
		$$D(z) = \{(x,y) \in \mathbb{R}^2 / x+y > 3\}$$
		\item 4.p
		\\$f(x,y,z)=\sqrt[•]{1-x^2}+\sqrt[•]{1-y^2}-\sqrt[•]{1-z^2}$
		\\$1-x^2\geq 0$   logo    $|x|^2 \leq 1$
		\\$1-y^2 \geq 0$  logo    $|y|^2 \leq 1$
		\\$1-z^2 \geq 0$  logo    $|z|^2 \leq 1$
		$$D(f)=\{(x,y,z) \in \mathbb{R}^3 / -1\leq x,y,z \leq 1\}$$
		\item 5.b
		\\$x^2+(y-3)^2+z^2 = 9$
		\begin{enumerate}
		\item$z_1=+\sqrt[•]{x^2+y^2-6y}$
		\item$z_2=-\sqrt[•]{x^2+y^2-6y}$
		\end{enumerate}
		$x^2+y^2 - 6y \geq 0$
		\\$D(z)=\{(x,y)\in \mathbb{R}^2, x^2+y^2 \geq 6y\}$
%		\item 7.a
%		\item 7.d	
%		\item 7.e
%		\item 8	
%		\item 9
%		\item 11
%		\item 13.a
%		\item 13.b
%		\item 13.c
%		\item 13.d
%		\item 13.g
%		\item 13.h
%		\item 13.j
%		\item 15(a)
%		\item 16
%		\item 17
	\end{itemize}
%
\section*{Lista 3.7 (pg. 99)}
	\begin{itemize}
		\item 1.a
		\\$x^2+y^2-2y<3$
		\\$(x-x_0)^2+(y-y_0)^2 = r^2$
		\\$(x-0)^2+(y-1)^2 = 4 \to x^2+y^2-2y=3$
		\\Bola aberta em centro $(0,1)$ e raio $2$.
		\item 1.b
		\\$x^2+y^2+z^2+6z<0$
		\\$(x-x_0)^2+(y-y_0)^2+(z-z_0)^2 = x^2+y^2+z^2+6z$
		\\$x_0 = 0$
		\\$y_0 = 0$
		\\$z_0 = -3$
		\\$r = 3$
		\\$x^2+y^2+(z+3)^2=3^2$
		\\Bola aberta em centro $(0,0,-3)$ e raio $3$.
		\item 1.e
		\\$x^2+y^2-1>0$
		\\$x^2+y^2>1$ não é uma bola.
		\item 1.f
		\\$x^2+4x+y^2<5$
		\\$(x+2)^2+(y-0)^2=r^2$
		\\$(x+2)^2+y^2=x^2+4x+y^2+4=r^2$
		\\$r^2-4<5$
		\\$r^2<9$
		\\$r<3$
		Bola aberta centrada em $(-2,0)$ e raio $3$.
		\item 2
		\\$A = \{(x,y) \in \mathbb{R}^2 / 2<x<3$ e $-1<y<1\}$
		\\Fronteira: $(2,-1)\to (3,-1) \to (3,1) \to (2,1) \to (2,-1)$
		\item 3
		\\$B=\{(x,y,z)\int \mathbb{R}^3/-1<x<1$, $-1<y<1$ e $-1<z<1\}$
		\\Fronteira: Cubo formado pelos vértices $(-1,-1,-1)$ até $(1,1,1)$
		\item 4 Identificar as afirmações verdadeiras:
			\begin{enumerate}
				\item A união de bolas abertas é uma bola.
				\item A união de bolas abertas é um conjunto aberto.
				\item A união de bolas abertas é um conjunto conexo.
				\item O conjunto $A = \{(x,y)/x^2+2x+y^2-4y>0\}$ é conexo.
				\item O conjunto $B = \{(x,y)/x^2>y^2\}$ é aberto.
			\end{enumerate}
			
%		\item 6(a,d)
%		\item 10
%		\item 13(a,b,e)
%		\item 14(a,c)
%		\item 15(a,e)
%		\item 17(b,d,f)
%		\item 18(a,b,d)
%		\item 19(a,c)
%		\item 20(a,b,c,d,e,f,g)
%		\item 21(a,c,e,i,n,p)
%		\item 22(a,b,e,g,i)
%		\item 23
%		\item 24
%		\item 25
	\end{itemize}
\end{document}
