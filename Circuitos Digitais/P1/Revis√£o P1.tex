\documentclass{article}

%\usepackage[brazil]{babel}

\title{Revisão Prova 1 de Circuitos Digitais}
\author{Erickson Giesel Müller}
\date{16 de Maio de 2024}

\begin{document}
	\maketitle
	\section{Conteúdos}
		\begin{enumerate}
			\item Algebra de Boole
			\item Circuitos, Tabela-Verdade e Expressões
			\item Conversão de Expressões Booleanas (Soma de Produtos e Produtos de Soma)
			\item Simplificação Algébrica
			\item Mapas de Karnaugh
		\end{enumerate}
		
	\section{Algebra de Boole}
	
		Algebra de Boole é a matemática dos circuitos digitais, calculada usando variáveis e seus valores, é através dela que podemos demonstrar o que acontece nas portas lógicas. Uma variável pode assumir o valor 0 ou 1. Se a Variável $A$ for 1, o complemento dessa variável será 0, denominado $A$ negado ou $\overline{A}$.
		\subsection{Adição Booleana}
			É o equivalente à porta lógica $OR$. Se um dos dois termos à serem somados for 1, o resultado será 1.\\
				\hspace*{4 cm}
				\begin{tabular}{|c|c|c|}
					\hline
					A & B & A+B \\
					\hline
					0 & 0 & 0 \\
					\hline
					0 & 1 & 1 \\
					\hline
					1 & 0 & 1 \\
					\hline
					1 & 1 & 1 \\
					\hline
				\end{tabular}
				
		\subsection{Diferença entre $\overline{A}+\overline{B}$ e $\overline{A+B}$}
			\hspace*{2 cm}
			\begin{tabular}{|c|c|c|}
				\hline
				A & B & $\overline{A}+\overline{B}$ \\
				\hline
				0 & 0 & 1 \\
				\hline
				0 & 1 & 1 \\
				\hline
				1 & 0 & 1 \\
				\hline
				1 & 1 & 0 \\
				\hline
			\end{tabular}
			\hspace*{2 cm}
			\begin{tabular}{|c|c|c|}
				\hline
				A & B & $\overline{A+B}$ \\
				\hline
				0 & 0 & 1 \\
				\hline
				0 & 1 & 0 \\
				\hline
				1 & 0 & 0 \\
				\hline
				1 & 1 & 0 \\
				\hline
			\end{tabular}
			
		\subsection{Multiplicação Booleana}
			A multiplicação é equivalente à porta $AND$ e o resultado será 1 quando todas as variáveis da multiplicação forem 1. A diferença entre $\overline{A}.\overline{B}$ e $\overline{A.B}$ é que $\overline{A}.\overline{B}$ será 1 quando $A$ e $B$ forem 0, já $\overline{A.B}$ será 1 quando $A$ e $B$ forem diferentes de 1.\\
			
			\begin{tabular}{|c|c|c|}
				\hline
				A & B & $A.B$ \\
				\hline
				0 & 0 & 0 \\
				\hline
				0 & 1 & 0 \\
				\hline
				1 & 0 & 0 \\
				\hline
				1 & 1 & 1 \\
				\hline
			\end{tabular}
			\hspace*{1 cm}
			\begin{tabular}{|c|c|c|}
				\hline
				A & B & $\overline{A}.\overline{B}$ \\
				\hline
				0 & 0 & 1 \\
				\hline
				0 & 1 & 0 \\
				\hline
				1 & 0 & 0 \\
				\hline
				1 & 1 & 0 \\
				\hline
			\end{tabular}			
			\hspace*{1 cm}
			\begin{tabular}{|c|c|c|}
				\hline
				A & B & $\overline{A.B}$ \\
				\hline
				0 & 0 & 1 \\
				\hline
				0 & 1 & 1 \\
				\hline
				1 & 0 & 1 \\
				\hline
				1 & 1 & 0 \\
				\hline
			\end{tabular}
			\\
			Podemos perceber que $\overline{A}+\overline{B}$ é igual a $\overline{A.B}$; e $\overline{A+B}$ é igual a $\overline{A}.\overline{B}$. Teorema de DeMorgan.
			
\end{document}