\documentclass[•]{article}
\usepackage[]{amsmath}
\title{Revisão P1 Pesquisa e Ordenação de Dados\\Giancarlo}
\author{Erickson G. Müller}

\begin{document}
	\maketitle
	\section{Conteúdos}
		\begin{enumerate}
			\item Complexidade de Algoritmos
			\item Bubble Sort
			\item Selection Sort
			\item Insertion Sort
			\item Merge Sort
			\item Quick Sort
			\item Heap Sort
		\end{enumerate}
	\newpage
	\section{Métodos de Ordenação}
		\subsection{Ordenação Estável}
			Preserva a ordem relativa dos elementos que possuem o mesmo valor para a chave de ordenação. Composto por mais de uma chave.
		\subsection{Ordenação Não Estável}
		\subsection{In place/In situ}
			Os valores são permutados dentro da própria estrutura do vetor, não havendo necessidade de duplicar a memória.

	\section{Merge Sort}	
		\begin{center}
			Dividir em elementos ordenados  e depois intercalar na ordem correta
		\end{center}
	\section{Quick Sort}
		Não precisa de memória extra(in-place).\\
		Em tese $n \log n$\\
		Pior caso = $n^2$(Quando já está ordenado).\\
		Algoritmo de \textbf{Ordenação Instável}\\
		Divisão e consquista.\\
		i = posição que estou fazendo a comparação\\
		k = 
		se $i>pivo \to$ k fica parado e i vai para o próximo elemento\\
		se $pivo>i \to$ elemento k troca com i, k e i vão para o próximo elemento\\
		Quando i chega na posição do pivô $\to$ trocar o k pelo pivô
		\subsection{regras}
		Se o Elemento for maior que o pivô	: i anda, k fica parado\\
		Se o Elemento for menor que o pivô: i anda, k anda\\
		Se o Elemento for menor que a posição do k: troca elemento com o k, k anda
		Passos:
		\begin{enumerate}
			\item Escolher o pivô (tradicionalmente o último elemento);
			\item Particionamento (posicionar em relação ao pivô)
			
			
		\end{enumerate}
\end{document}
